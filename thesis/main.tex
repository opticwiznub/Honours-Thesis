%%%%%%%%%%%%%%%%%%%%%%%%%%%%%%%%%%%%%%%%%%%%%%%%%%%%%%%%%%%%%%%%%%%%%%%
%
%  A small sample UNSW Honours Thesis file.
%  Any questions to Ian Doust i.doust@unsw.edu.au
%
% Edited CSG 11.9.2015, use some of Gery's ideas for front matter; add a conclusion chapter.
%%%%%%%%%%%%%%%%%%%%%%%%%%%%%%%%%%%%%%%%%%%%%%%%%%%%%%%%%%%%%%%%%%%%%%%
%
%  The first part pulls in a UNSW Thesis class file.  This one is
%  slightly nonstandard and has been set up to do a couple of
%  things automatically
%
 
\documentclass[honours,12pt]{unswthesis}
\linespread{1}
\usepackage{amsfonts}
\usepackage{amssymb}
\usepackage{amsthm}
\usepackage{latexsym,amsmath}
\usepackage{graphicx}
\usepackage{afterpage}

%%%%%%%%%%%%%%%%%%%%%%%%%%%%%%%%%%%%%%%%%%%%%%%%%%%%%%%%%%%%%%%%%
%
%  The following are some simple LaTeX macros to give some
%  commonly used letters in funny fonts. You may need more or less of
%  these
%
\newcommand{\R}{\mathbb{R}}
\newcommand{\Q}{\mathbb{Q}}
\newcommand{\C}{\mathbb{C}}
\newcommand{\N}{\mathbb{N}}
\newcommand{\F}{\mathbb{F}}
\newcommand{\PP}{\mathbb{P}}
\newcommand{\T}{\mathbb{T}}
\newcommand{\Z}{\mathbb{Z}}
\newcommand{\B}{\mathfrak{B}}
\newcommand{\BB}{\mathcal{B}}
\newcommand{\M}{\mathfrak{M}}
\newcommand{\X}{\mathfrak{X}}
\newcommand{\Y}{\mathfrak{Y}}
\newcommand{\CC}{\mathcal{C}}
\newcommand{\E}{\mathbb{E}}
\newcommand{\cP}{\mathcal{P}}
\newcommand{\cS}{\mathcal{S}}
\newcommand{\A}{\mathcal{A}}
\newcommand{\ZZ}{\mathcal{Z}}
%%%%%%%%%%%%%%%%%%%%%%%%%%%%%%%%%%%%%%%%%%%%%%%%%%%%%%%%%%%%%%%%%%%%%
%
% The following are much more esoteric commands that I have left in
% so that this file still processes. Use or delete as you see fit
%
\newcommand{\bv}[1]{\mbox{BV($#1$)}}
\newcommand{\comb}[2]{\left(\!\!\!\begin{array}{c}#1\\#2\end{array}\!\!\!\right)
}
\newcommand{\Lat}{{\rm Lat}}
\newcommand{\var}{\mathop{\rm var}}
\newcommand{\Pt}{{\mathcal P}}
\def\tr(#1){{\rm trace}(#1)}
\def\Exp(#1){{\mathbb E}(#1)}
\def\Exps(#1){{\mathbb E}\sparen(#1)}
\newcommand{\floor}[1]{\left\lfloor #1 \right\rfloor}
\newcommand{\ceil}[1]{\left\lceil #1 \right\rceil}
\newcommand{\hatt}[1]{\widehat #1}
\newcommand{\modeq}[3]{#1 \equiv #2 \,(\text{mod}\, #3)}
\newcommand{\rmod}{\,\mathrm{mod}\,}
\newcommand{\p}{\hphantom{+}}
\newcommand{\vect}[1]{\mbox{\boldmath $ #1 $}}
\newcommand{\reff}[2]{\ref{#1}.\ref{#2}}
\newcommand{\psum}[2]{\sum_{#1}^{#2}\!\!\!'\,\,}
\newcommand{\bin}[2]{\left( \begin{array}{@{}c@{}}
				#1 \\ #2
			\end{array}\right)	}
%
%  Macros - some of these are in plain TeX (gasp!)
%
\newcommand{\be}{($\beta$)}
\newcommand{\eqp}{\mathrel{{=}_p}}
\newcommand{\ltp}{\mathrel{{\prec}_p}}
\newcommand{\lep}{\mathrel{{\preceq}_p}}
\def\brack#1{\left \{ #1 \right \}}
\def\bul{$\bullet$\ }
\def\cl{{\rm cl}}
\let\del=\partial
\def\enditem{\par\smallskip\noindent}
\def\implies{\Rightarrow}
\def\inpr#1,#2{\t \hbox{\langle #1 , #2 \rangle} \t}
\def\ip<#1,#2>{\langle #1,#2 \rangle}
\def\lp{\ell^p}
\def\maxb#1{\max \brack{#1}}
\def\minb#1{\min \brack{#1}}
\def\mod#1{\left \vert #1 \right \vert}
\def\norm#1{\left \Vert #1 \right \Vert}
\def\paren(#1){\left( #1 \right)}
\def\qed{\hfill \hbox{$\Box$} \smallskip}
\def\sbrack#1{\Bigl \{ #1 \Bigr \} }
\def\ssbrack#1{ \{ #1 \} }
\def\smod#1{\Bigl \vert #1 \Bigr \vert}
\def\smmod#1{\bigl \vert #1 \bigr \vert}
\def\ssmod#1{\vert #1 \vert}
\def\sspmod#1{\vert\, #1 \, \vert}
\def\snorm#1{\Bigl \Vert #1 \Bigr \Vert}
\def\ssnorm#1{\Vert #1 \Vert}
\def\sparen(#1){\Bigl ( #1 \Bigr )}

\newcommand\blankpage{%
    \null
    \thispagestyle{empty}%
    \addtocounter{page}{-1}%
    \newpage}

%%%%%%%%%%%%%%%%%%%%%%%%%%%%%%%%%%%%%%%%%%%%%%%%%%%%%%%%%%%%%%
%
% These environments allow you to get nice numbered headings
%  for your Theorems, Definitions etc.  
%
%  Environments
%
%%%%%%%%%%%%%%%%%%%%%%%%%%%%%%%

\newtheorem{theorem}{Theorem}[section]
\newtheorem{lemma}[theorem]{Lemma}
\newtheorem{proposition}[theorem]{Proposition}
\newtheorem{corollary}[theorem]{Corollary}
\newtheorem{conjecture}[theorem]{Conjecture}
\newtheorem{definition}[theorem]{Definition}
\newtheorem{example}[theorem]{Example}
\newtheorem{remark}[theorem]{Remark}
\newtheorem{question}[theorem]{Question}
\newtheorem{notation}[theorem]{Notation}
\numberwithin{equation}{section}

%%%%%%%%%%%%%%%%%%%%%%%%%%%%%%%%%%%%%%%%%%%%%%%%%%%%%%%%%%%%%%%%%%
%
%  If you've got some funny special words that LaTeX might not
% hyphenate properly, you can give it a helping hand:
%
\hyphenation{Mar-cin-kie-wicz Rade-macher}

%%%%%%%%%%%%%%%%%%%%%%%%%%%%%%%%%%%%%%%%%%%%%%%%%%%%%%%%%%%%%%%%%%
% 
% OK...Now we get to some actual input.  The first part sets up
% the title etc that will appear on the front page
%
%%%%%%%%%%%%%%%%%%%%%%%%%%%%%%%%%%%%%%%%%%%%%%%%%%%%%%%%%%%%%%%%%

\title{Title?}

\authornameonly{Justin Clarke}

\author{\Authornameonly\\{\bigskip}Supervisor: Associate Professor Yanan Fan}

\copyrightfalse
\figurespagefalse
\tablespagefalse

%%%%%%%%%%%%%%%%%%%%%%%%%%%%%%%%%%%%%%%%%%%%%%%%%%%%%%%%%%%%%%%%%
%
%  And now the document begins
%  The \beforepreface and \afterpreface commands puts the
%  contents page etc in
%
%%%%%%%%%%%%%%%%%%%%%%%%%%%%%%%%%%%%%%%%%%%%%%%%%%%%%%%%%%%%%%%%%%

\begin{document}

\beforepreface

\afterpage{\blankpage}

% plagiarism

\prefacesection{Plagiarism statement}

\vskip 10pc \noindent I declare that this thesis is my
own work, except where acknowledged, and has not been submitted for
academic credit elsewhere. 

\vskip 2pc  \noindent I acknowledge that the assessor of this
thesis may, for the purpose of assessing it:
\begin{itemize}
\item Reproduce it and provide a copy to another member of the University; and/or,
\item Communicate a copy of it to a plagiarism checking service (which may then retain a copy of it on its database for the purpose of future plagiarism checking).
\end{itemize}

\vskip 2pc \noindent I certify that I have read and understood the University Rules in
respect of Student Academic Misconduct, and am aware of any potential plagiarism penalties which may 
apply.\vspace{24pt}

\vskip 2pc \noindent By signing 
this declaration I am
agreeing to the statements and conditions above.
\vskip 2pc \noindent
Signed: \rule{7cm}{0.25pt} \hfill Date: \rule{4cm}{0.25pt} \newline
\vskip 1pc

\afterpage{\blankpage}

% Acknowledgements are optional


\prefacesection{Acknowledgements}

% {\bigskip}By far the greatest thanks must go to my supervisor for
% the guidance, care and support they provided. 

% {\bigskip\noindent}Thanks 
% must also go to Emily, Michelle, John and Alex who helped by
% proof-reading the document in the final stages of preparation.

% {\bigskip\noindent}Although
% I have not lived with them for a number of years, my family also deserve
% many thanks for their encouragement.

% {\bigskip\noindent} Thanks go to Robert Taggart for allowing his thesis
% style to be shamelessly copied.

% {\bigskip\bigskip\bigskip\noindent} Fred Flintstone, 2 November 2015.

\afterpage{\blankpage}

% Abstract

\prefacesection{Abstract}

Graph sparification techniques for graph neural networks have traditionally been used to accelerate training and inference on real-world graphs which have billions of paramaters.
There are also many different climate models which use complex mathematical models to model the interactions between energy and matter over the world. 
Many of these models share components and parameters and in this paper, I attempt to quantify these relationships through graph sparsification. (Talk more about climate?)
\afterpage{\blankpage}


\afterpreface

%%%%%%%%%%%%%%%%%%%%%%%%%%%%%%%%%%%%%%%%%%%%%%%%%%%%%%%%%%%%%%%%%%
%
% Now we can start on the first chapter
% Within chapters we have sections, subsections and so forth
%
%%%%%%%%%%%%%%%%%%%%%%%%%%%%%%%%%%%%%%%%%%%%%%%%%%%%%%%%%%%%%%%%%%

\afterpage{\blankpage}

\chapter{Introduction}\label{s-intro}


{\noindent} The simplest climate modeles have existed since the 1950's with the very first computers modelling small two-dimensional climates. 
Modern models have become increasingly more complex in part due to the increasing computational power available today and the large amount of data available
worldwide to train these models on. (Talk more about the math behind these models? Stochastic Differential models or talk about a few of the main models in use today?)

{\noindent} In recent years, Graph neural networks have become the premier method of processing data with non-cartesian structure. Much of this data exists in the world
in applications such as chemical analysis, social networks and link prediction (Insert references for each from reading).       

{\section{Problem Formulation}}\label{problem formulation}
We define a graph as ${\mathcal{G}} = ({\mathcal{V}}, \textbf{A})$, where $\mathcal{V}$ represents a set of verticies which contains a list of nodes
$\{ v_1, \dots, v_n \}$ and $\textbf{A} \in \mathbb{R}^{n \times n}$ the adjacentcy matrix which contains 


% Given a sequence $\{x_n\}_{n=1}^{\infty}$ in a Banach space $\X$,
% the following natural questions may be asked.
% \begin{enumerate}
% \item[(Q1)] Which elements of the space $\X$ can be expressed in the form
% $\sum_{n=1}^{\infty}a_nx_n$ for some unique sequence $\{a_n\}_{n=1}^{\infty}$
% of scalars?

% \item[(Q2)] Supposing $x\in\X$ can be written as $\sum_{n=1}^{\infty}a_nx_n$,
% does it matter in what order the terms of the series are summed?

% \item[(Q3)] In the setting of (Q2), will the series
% $\sum_{n=1}^{\infty}\varepsilon_na_nx_n$ converge for any choice of
% $\varepsilon_n=\pm 1$?

% \item[(Q4)] Given a scalar sequence $\{\phi_n\}_{n=1}^{\infty}$, does it give
% rise to a bounded linear mapping
% $\sum_{n=1}^{\infty}a_nx_n\mapsto\sum_{n=1}^{\infty}\phi_na_nx_n$?
% \end{enumerate}

% {\noindent}Answering these and similar questions for
% function spaces whose members act on the circle group $\T\cong[0,2\pi]$
% is the main subject of this thesis.

% In Chapter 1, we give a general picture of the situation, introducing the
% concepts of Schauder bases and conditional convergence of sequences in Banach
% spaces. Standard examples show that not every Banach space with a basis
% $\{x_n\}_{n=1}^{\infty}$ has the property that the terms in each expansion 
% $\sum_{n=1}^{\infty}a_nx_n$ can be freely rearranged.


% The above questions have been the subject of much study for the trigonometric
% sequence $\{x_n\}_{n=-\infty}^{\infty}$ contained in $\X=L^p(\T)$ and given by
% $x_n(t)=e^{int}$. If $1<p<\infty$, then it is known that
% $\{x_n\}_{n=-\infty}^{\infty}$ is a basis for $\X$. Moreover, each $f\in\X$ has
% the expansion
% \begin{equation}\label{intro1}
% f=\sum_{n=-\infty}^{\infty}a_nx_n
% \end{equation}
% where the series converges in $\X$ and for each $n\in\Z$,
% \begin{equation}\label{intro2}
% a_n=\frac{1}{2\pi}\int_0^{2\pi}e^{-ins}f(s)\,ds.
% \end{equation}
% In Chapter 2 we give an account of this theory in as much as it applies to
% answering questions similar to those above. In particular, we recall three
% classical results from harmonic analysis --- the M. Riesz Conjugacy Theorem,
% the Littlewood--Paley Theorem and the Strong Marcinkiewicz Multiplier Theorem.
% These demonstrate that the expansion (\ref{intro1}) is valid for each 
% $f\in L^p(\T)$ (for $1<p<\infty$) and give sequences
% $\{\phi_n\}_{n=1}^{\infty}$ for which (Q4) has a positive answer.

% In Chapter 3 we present some results of Bourgain
% which generalise these classical theorems to the $L^p(\T,\Y)$ spaces
% --- those $L^p$ spaces whose functions take values in a Banach space $\Y$. 
% When
% $\Y$ is a Banach space that has the so-called UMD property and when 
% $1<p<\infty$,
% analogues of the M. Riesz, Littlewood--Paley and Marcinkiewicz Theorems hold. In
% particular, equation (\ref{intro1}) is valid if $f$ belongs to any of these
% function spaces.

% After examining these vector-valued function spaces, we look at functions on
% $\T$ which are operator-valued. A certain class of functions, whose members are
% strongly continuous group homorphisms of $\T$ into $\B(\Y)$, yields analogues of
% the M. Riesz, Littlewood--Paley and Marcinkiewicz Theorems if
% $\Y$ has the UMD property.
% These analogues were first announced in the 1980s and 1990s by Berkson
% and Gillespie (see \cite{BG Fourier} and \cite{BG Spectral}). They are stated
% in Chapter 4, where it is also shown how they
% specialise to the classical theorems. In particular, given a strongly continuous
% representation $R:\T\rightarrow\B(\Y)$, its `Fourier series'
% \[R=\sum_{n=-\infty}^{\infty}P_nx_n,\]
% converges. The $n$th `Fourier coefficient' $P_n$ of $R$ is a projection
% on $\Y$ given by the formula
% \[P_n=\frac{1}{2\pi}\int_0^{2\pi}e^{ins}R(s)\,ds.\]
% The similarity between the above two equations and equations (\ref{intro1}) and
% (\ref{intro2}) is striking.

% Proving the theorems stated in Chapter 4 is the main objective of 
% Chapters 5 and 6. Contained in the proofs are a wide range of techniques
% taken from harmonic analysis and Banach space operator theory. 
% The parts of the proofs of the two main theorems are dispersed across the 
% literature with significant portions being found in
% \cite{BBG}, \cite{BG Fourier}, \cite{BG Spectral}, \cite{BGM} and \cite{Bourg}.
% These chapters provide what is most likely the only unified account of these
% proofs. Chapter 6 ends
% with an example that demonstrates the scope of these theorems. Applying the
% Strong Marcinkiewicz analogue for strongly continuous representations, a wide
% class of multiplier
% projections acting on the von Neumann Schatten $\CC_p$ spaces are given. This
% class was recently discovered by Doust and Gillespie (see \cite{DG}).

% This thesis is a coherent presentation of a quest to generalise three classical
% theorems that were discovered in the 1920s, 1930s and 1940s. Their analogues are
% the product of a conglomeration of ideas that straddle the 1980s and 1990s and
% the application of these new results brings the story into the twenty-first
% century.



%%%%%%%%%%%%%%%%%%%%%%%%%%%%%%%%%%%%%

\chapter{Convergence and Bases on Banach Spaces}

%%%%%%%%%%%%%%%%%%%%%%%%%%%%%%%%%%%%%


In this chapter we outline the general context in which our study of multiplier
theory in Banach spaces
takes place. We begin by revising the elementary concepts
associated with functional analysis in Banach spaces. Then Section \ref{bases}
introduces the notions of bases, multiplier transforms and unconditional
convergence of sequences, setting the scene for the rest of the thesis. The 
chapter ends by introducing a technique which is frequently used to obtain 
results pertaining to the unconditionality of sequences.


%%%%%%%%%%%
\section{Banach, Operator and Dual Spaces}\label{Banach Spaces}
%%%%%%%%%%%

We recall some basic definitions and concepts from Banach space theory and
functional analysis. A good introductory reference for this material is 
\cite{Con}.

A {\em Banach space} $\X$ is a vector space $V$ over a field $\F$ that is 
equipped with a norm
$\norm{\,\cdot\,}_{\X}$ that makes $V$ complete with respect to the metric $d$
given by $d(x,x')=\norm{x-x'}_{\X}$. In this thesis we shall always assume that
the
underlying field $\F$ of scalars is $\C$. Suppose $\Y$ is also a Banach space
with norm $\norm{\,\cdot\,}_{\Y}$. A linear operator
$T:\X\rightarrow\Y$ is said to be {\em bounded} if
$\sup\{\norm{Tx}_{\Y}:x\in\X,\norm{x}_{\X}=1\}<\infty$. The collection of
$\Y$-valued bounded linear operators on $\X$ is denoted $\B(\X,\Y)$, and is
itself a Banach space when equipped with the {\em operator norm}
$\norm{T}=\sup\{\norm{Tx}_{\Y}:x\in\X,\norm{x}_{\X}\leq 1\}$. A $\Y$-valued
linear
operator on $\X$ is bounded if and only if it is continuous with respect to the
topologies on $\X$ and $\Y$ induced by their respective norms. Furthermore, if
$T\in\B(\X,\Y)$ then $\norm{Tx}_{\Y}\leq\norm{T}\norm{x}_{\X}$ for all $x\in\X$.

Henceforth, the norm of a Banach space $\X$ will be denoted
$\norm{\,\cdot\,}_{\X}$ (except when $\X$ is an $L^p$ space --- see the
example below), while the operator norm $\norm{\,\cdot\,}$ will
contain no subscript because it is clear from the operator's definition what
spaces its norm is induced from. The space $\B(\X,\X)$ of bounded linear
operators sending elements of $\X$ into $\X$ will be denoted $\B(\X)$. An
important class of operators included in $\B(\X)$ are the {\em projections} on
$\X$, those bounded linear maps $P$ satisfying $P^2=P$.

There are many examples of Banach spaces. We give one example that will feature
regularly throughout the thesis.

\begin{example}\label{L^p example}
Let $(\Omega,\A,\mu)$ be a measure space and $1\leq p<\infty$. Let
$\X=L^p(\Omega,\A,\mu)$ be the space of all equivalence classes of
$\C$-valued $\A$-measurable functions $f$ on $\Omega$ with the property that
$\int_{\Omega}|f(x)|^p\,d\mu(x)<\infty$. We shall usually blur the distinction
between the equivalence classes of $L^p(\Omega,\A,\mu)$ and the representatives
of these classes. The norm on $\X$ is given by
$\norm{f}_{\X}=(\int_{\Omega}|f(x)|^p\,d\mu(x))^{1/p}$ and
makes $\X$ a Banach space. When the underlying $\sigma$-algebra and measure is
standard (for example, when $\Omega=\R$, $\A$ is the Borel $\sigma$-algebra
and $\mu$ is Lebesgue measure on $\R$), we shall denote
$L^p(\Omega,\A,\mu)$ by $L^p(\Omega)$. 
Because it
is notationally less cumbersome to write $\norm{\,\cdot\,}_p$ instead of
$\norm{\,\cdot\,}_{L^p(\Omega,\A,\mu)}$, we shall do so whenever the context
permits no ambiguity. A particularly useful fact is that if $(\Omega,\A,\mu)$ is
a finite measure space and $1\leq p<r<q\leq\infty$, then
$L^p(\Omega)\cap L^q(\Omega)\subset L^r(\Omega)$ and
$L^q(\Omega)\subset L^p(\Omega)$. A quick source of information on these spaces
is \cite[\S 6.4]{Pedersen}.
\end{example}

\begin{example}
We give a few of the sequence spaces. 
\begin{enumerate}
\item For $1\leq p<\infty$, define $\lp$ to be
the space of all functions $f:\N\rightarrow\C$ such that
$\sum_{n=1}^{\infty}|f(n)|^p<\infty$, and equip it with the norm
$\norm{f}_{\lp}=(\,\sum_{n=1}^{\infty}|f(n)|^p\,)^{1/p}$. Thus $\lp$ may be
identified with $L^p(\N)$, where the underlying measure is counting measure on
$\N$.
\item The space $\ell^{\infty}$ is the set of all bounded functions
$f:\N\rightarrow\C$ with norm $\norm{f}_{\ell^{\infty}}=\sup\{|f(n)|:n\in\N\}$.
\item The subset $c$ of $\ell^{\infty}$ of all convergent sequences 
has norm given by $\norm{f}_c=\norm{f}_{\ell^{\infty}}$ for $f\in c$.
\item The subset $c_0$ of $\ell^{\infty}$ of all bounded functions converging to
zero has norm given by $\norm{f}_{c_0}=\norm{f}_{\ell^{\infty}}$ for $f\in c_0$.
\end{enumerate}These are all Banach spaces.
\end{example}

To every Banach space is associated another Banach space known as its dual.
Suppose $\X$ is a Banach space. A {\em linear functional} on $\X$ is a linear
map from $\X$ into $\C$. The set of all bounded linear functionals on $\X$,
denoted
by $\X^*$, is made a vector space via pointwise operations. We equip $\X^*$ with
a norm given by $\norm{x^*}_{\X^*}=\sup\{|x^*(x)|:x\in\X,\norm{x}\leq 1\}$.
Thus $\X^*$ coincides with $\B(\X,\C)$ and is itself a Banach
space. We call $\X^*$ the {\em dual space} of $\X$. Given $x^*\in\X^*$ and
$x\in\X$, it is standard to write
$\ip<x,x^*>$ for $x^*(x)$.

It can be shown that for $1<p<\infty$, the dual space of $\X=L^p(\Omega,\A,\mu)$
is isometrically isomorphic to $\X=L^q(X,\Omega,\mu)$, where $q$ satisfies
$1/p+1/q=1$. Thus $(\lp)^*$ is isomorphic to $\ell^q$ as a Banach space. It is
also known that $c_0^*=\ell^1$ and $(\ell^1)^*=\ell^{\infty}$. See
\cite[Chapter III, \S 5 and \S 11]{Con} for more details.

Given a Banach space $\X$, we may take its dual $\X^*$. This is also a Banach
space and hence has a dual $(\X^*)^*$, called {\em the second dual of $\X$}
and written $\X^{**}$. Continuing in this way we may construct a sequence
$\X,\X^*,\X^{**},\X^{***},\ldots$ of Banach spaces. For which
spaces $\X$ does this list contain any repetitions? This question leads us to
consider an important class of Banach spaces, known as the reflexive Banach
spaces.

Let $\X$ be Banach space. To each $x\in\X$ we may associate a unique element
$\hatt{x}\in\X^{**}$ by the rule $\hatt{x}(x^*)=\ip<x,x^*>$ for all $x^*\in\X$.
The map $x\mapsto\hatt{x}$ from $\X$ into $\X^{**}$ is called the {\em natural
map} of $\X$ into its second dual.

A Banach space $\X$ is said to be {\em reflexive} if
$\X^{**}=\{\hatt{x}:x\in\X\}$. If $\X$ is reflexive then its second dual
$\X^{**}$ is isometrically isomorphic to $\X$. However, the converse does not
hold (see \cite[III.11]{Con}). It is a consequence of the Riesz Representation
Theorem \cite[I.3.4]{Con} that every Hilbert space is reflexive.
The spaces $\X=L^p(X,\Omega,\mu)$ and
$\lp$ are also reflexive for $1<p<\infty$, but by the above remarks $c_0$ is 
not. Since
$c_0^{**}=\ell^{\infty}$, it is clear that $c_0\subset c_0^{**}$. In fact,
$c_0$ is isometrically embedded into its second dual $\ell^{\infty}$. This fact
generalises to
all Banach spaces. The space $C[0,1]$ of continuous functions on $[0,1]$ with
supremum norm is another example of a non-reflexive Banach space.

A {\em Banach algebra} $\mathfrak{A}$ is an algebra over $\F$ that has a norm
$\norm{\,\cdot\,}_{\mathfrak{A}}$ relative to which $\mathfrak{A}$ is a Banach
space and
\[\norm{ab}_{\mathfrak{A}}\leq\norm{a}_{\mathfrak{A}}\norm{b}_{\mathfrak{A}}\]
for all $a,b\in\mathfrak{A}$. If $\mathfrak{A}$ has an identity $e$, then it is
assumed that $\norm{e}_{\mathfrak{A}}=1$. In this thesis we shall always take
$\F$ to be $\C$. An example of a Banach algebra is the space $C[0,1]$ equipped
with the supremum norm, with multiplication of elements in $C[0,1]$ defined
pointwise.


%%%%%%%%%%%%
\section{Topologies in Banach Spaces}\label{topologies}
%%%%%%%%%%%%

A Banach space $\X$ is easily made a topological space with the topology induced
by its norm, called the {\em norm topology} of $\X$. We say a sequence of
vectors {\em converges in $\X$} if it converges in the norm topology of $\X$.
Another
topology is the {\em weak topology} of $\X$. A sequence $\{x_n\}_{n=1}^{\infty}$
of vectors in $\X$ converges to $x\in\X$ in the weak topology of $\X$ if
$\lim_{n\rightarrow\infty}\ip<x_n-x,x^*>=0$ for all $x^*\in\X^*$. As their names
suggest, convergence in the norm topology implies convergence in the weak
topology, or in other words, if a sequence converges then it also converges
weakly.

We shall consider three topologies in the Banach space $\B(\X)$.
Suppose $\{T_n\}_{n=1}^{\infty}$ is a sequence of operators in $\B(\X)$.
We say it
converges to $T$ {\em in norm} if $\norm{T-T_n}\rightarrow 0$, {\em strongly} if
$\norm{Tx-T_nx}_{\X}\rightarrow 0$ for all $x\in\X$, and {\em weakly} if
$\ip<Tx-T_nx,x^*>\rightarrow 0$ for all $x\in\X$ and $x^*\in\X^*$. The 
topologies
induced are called the norm operator topology, the strong operator topology and
the weak
operator topology of $\B(\X)$ respectively. A sequence which converges strongly
will converge weakly, and in turn, convergence in the norm topology implies
strong convergence. All three topologies have unique limits and distinguish
points.

The following proposition is easy to verify and is included for familiarisation
with the concepts discussed thus far. It shall also be used to establish later
results.

\begin{proposition}\label{SOT convergence}
Let $\X$ be a Banach space and suppose $\{T_n\}_{n=1}^{\infty}$ is a sequence of
operators uniformly bounded in norm by some $K>0$ and converging to $T$
in the strong operator topology of $\B(\X)$. Then $\norm{T}\leq
K$.
\end{proposition}

\begin{proof}
We want to show $\norm{Tx}_{\X}\leq K\norm{x}_{\X}$ for all $x\in\X$. Fix
$x\in\X$ and $\epsilon>0$ and choose an $n\in\N$ such that
$\norm{Tx-T_nx}_{\X}<\epsilon$. Then 
\[\norm{Tx}_{\X}\leq\norm{Tx-T_nx}_{\X}+\norm{T_nx}_{\X}
<\epsilon+K\norm{x}_{\X},\]
and the proposition follows.
\end{proof}



%%%%%%%%%%%%
\section{Bases in Banach spaces}\label{bases}
%%%%%%%%%%%%


The definition and usefulness of a basis in a finite dimensional vector space is
well-known. It is natural then to want an analogous concept for
Banach spaces. The most useful approach is found in the notion of a Schauder
basis. A comprehensive introduction to such bases is \cite[pp. 1--52]{Lind},
upon which most of the material in this section is based. 

\begin{definition}
A sequence $\{x_n\}_{n=1}^{\infty}$ in a Banach space $\X$ is called a {\em 
Schauder basis of $\X$} if, for every $x\in\X$, there is a unique sequence of
scalars $\{a_n\}_{n=1}^{\infty}$ such that $x=\sum_{n=1}^{\infty}a_nx_n$. A
sequence $\{x_n\}_{n=1}^{\infty}$ which is a Schauder basis for its closed
linear span is called a {\em basic sequence}.
\end{definition}

A Banach space $\X$ is called {\em separable} if it has a countable dense 
subset.
Thus if $\X$ has a Schauder basis then $\X$ is separable. In general the 
converse
is not true. Since the class of Schauder bases is the only type of bases
considered in this
thesis, it will henceforth be implicit that any discussion about bases refers
only to Schauder bases.


\begin{example}
The ordered set of unit vectors $e_n=(0,0,0,\ldots,1,0,\ldots)$, where the $1$
occurs in the $n$th coordinate, forms a basis in each of the spaces
$c$, $c_0$ and $\lp$. Another example of a basis in $c$ is given by
$x_1=(1,1,1,\ldots)$
and $x_n=e_{n-1}$ for $n>1$. If $x=(b_1,b_2,\ldots)\in c$, then the expansion of
$x$ with respect to this basis is
\[x=(\lim_{k\rightarrow\infty}a_k)x_1+(a_1-\lim_{k\rightarrow\infty}a_k)x_2+
(a_2-\lim_{k\rightarrow\infty}a_k)x_3+\cdots.\] 
\end{example}


If a Banach space $\X$ has a basis, we may consider $\X$ as a sequence
space via an identification of each element $x=\sum_{n=1}^{\infty}a_nx_n$ in
$\X$ with the unique sequence of coefficients $(a_1,a_2,a_3,\ldots)$. To do this
note that it is essential to describe the basis as an ordered sequence rather
than merely as a set. The following example highlights a deeper reason for
why we must specify the ordering of the vectors in a basis.

\begin{example}\label{summing basis}
Consider the sequence of vectors $\{x_n\}_{n=1}^{\infty}$ in $c$ given by
\[x_n=(\underbrace{0,0,\ldots,0}_{n-1},1,1,\cdots),\]
for $n\in\Z^+$. It is not hard to see that this is a basis for $c$.
It is called the {\em summing basis} in $c$.
Consider the sequence $b$ defined by
\[b_n=\sum_{k=1}^n(-1)^{k+1}\frac{1}{k}\]
for $n\in\Z^+$. By Leibniz' test for alternating series,
$b_n$ has a limit as $n\rightarrow\infty$, so $b\in c$. The expansion of $b$ 
with
respect to the summing basis is $b=\sum_{n=1}^{\infty}a_nx_n$ where
$a_n=(-1)^{n+1}\frac{1}{n}$. Now consider the following rearrangement of the
series. Define $\pi:\Z^+\rightarrow\Z^+$ by
\[\pi(n)=\left\{\begin{array}{ll}
				2k & \mbox{if $n=3k$}\\
				4k+1 & \mbox{if $n=3k+1$}\\
				4k+3 & \mbox{if $n=3k+2$.}
			\end{array}\right.
\]
Then $\pi$ is clearly a permutation of $\Z^+$. However, the expansion
\[b'\equiv\sum_{n=1}^{\infty}a_{\pi(n)}x_{\pi(n)}\]
does not lie in $c$ since
\[b_{3n}'=\sum_{k=1}^n\left(\frac{1}{2k-1}+\frac{1}{2k+1}-\frac{1}{2k}\right)
\geq 2\sum_{k=1}^n\frac{1}{k}\]
diverges as $n\rightarrow\infty$.
\end{example}

Not every basis has the defect that the convergence of an expansion with respect
to the basis is dependent on the order of summation of the expansion. For
example, expansions in finite dimensional spaces can be summed in any order
without altering convergence. The following proposition gives various methods 
for
checking whether a sum of vectors can be freely rearranged while still
respecting convergence.

\begin{proposition}\label{unconditional conv}
\cite[Proposition 1.c.1]{Lind}
Let $\{x_n\}_{n=1}^{\infty}$ be a sequence of vectors in a Banach space $\X$.
Then the following are equivalent.

(i) The series $\sum_{n=1}^{\infty}x_{\pi(n)}$ converges for every permutation
$\pi$ of the integers.

(ii) The series $\sum_{k=1}^{\infty}x_{n_k}$ converges for every choice of
$n_1<n_2<n_3\ldots$.

(iii) The series $\sum_{n=1}^{\infty}\varepsilon_nx_n$ converges
for every choice of signs $\varepsilon_n=\pm 1$.

(iv) For every $\epsilon>0$ there exists an integer $n_0$ such that
$\ssnorm{\sum_{j\in J}x_j}_{\X}<\epsilon$ for every finite set of integers $J$
which satisfies $\min\{j\in J\}>n_0$.
\end{proposition}

With the aid of the proposition, there are now easier ways to verify that the
series
$\sum_{n=1}^{\infty}a_nx_n$ given in Example \reff{bases}{summing basis} is
dependent on the order of summation. Simply observe that
$\sum_{n=1}^{\infty}a_{2n-1}x_{2n-1}$ does not converge in $c$ and hence 
statement (ii) of the proposition fails to hold.

\begin{definition}
Let $\{x_n\}_{n=1}^{\infty}$ be a sequence of vectors in a Banach space $\X$.
A series $\sum_{n=1}^{\infty}x_n$ which satisfies any of the conditions (i),
(ii), (iii) or (iv) in Proposition \reff{bases}{unconditional conv} is said to 
be
{\em unconditionally convergent}. A basis $\{x_n\}_{n=1}^{\infty}$ of a Banach
space $\X$ is {\em unconditional} if for every $x\in\X$, its expansion
$x=\sum_{n=1}^{\infty}a_nx_n$ converges unconditionally. Otherwise the basis is
said to be {\em conditional}.
\end{definition}

Thus the standard ordered bases $\{e_n\}_{n=1}^{\infty}$ for $c$, $c_0$ and
$\lp$,
$1\leq p<\infty$ are unconditional bases. This means that any reordering of
$\{e_n\}_{n=1}^{\infty}$ also yields unconditional bases in these spaces.
The summing basis for $c$ is clearly a conditional basis.

Given a basis $\{x_n\}_{n=1}^{\infty}$, one might ask what
complex sequences $\{\phi_n\}_{n=1}^{\infty}$ give rise to a bounded multiplier
transform? That is, is there a constant $C>0$ such that for all $x\in\X$,
\[\snorm{\sum_{n=1}^{\infty}\phi_na_{n,x}x_n}_{\X}\leq C\norm{x}_{\X},\]
where $\sum_{n=1}^{\infty}a_{n,x}x_n$ is the expansion of each $x\in\X$ with
respect to the basis? If that basis is unconditional, then the space of such
sequences is just $\ell^{\infty}$ (see \cite[Proposition 1.c.7]{Lind}).


We have seen that conditional bases do exist, and some of the resulting
difficulties that arise when working with a conditional basis. Is there any way
in which we can avoid such problems in separable Banach spaces? In other words,
can we always find an unconditional
basis for a separable Banach space? The answer is no. The space $L^1[0,1]$ has
no unconditional basis. In fact, $L^1[0,1]$ cannot even be embedded in a space
with an unconditional basis \cite[Propostion 1.d.1]{Lind}.

There are many open problems regarding unconditional bases. For example,
suppose $\X$ is a Banach space with an unconditional basis, and let $\Y$ be
a complemented subspace of $\X$. Does $\Y$ have an unconditional basis? One
outstanding problem has recently been solved (see \cite{Gowers}).
Does every infinite dimensional Banach space $\X$ contain an unconditional
basic sequence? The answer is no.
We shall not pursue such broad problems in this thesis. Instead, we shall use 
tools from harmonic analysis and spectral theory to examine whether or not
particular sequences in certain Banach spaces are conditional, and whether or 
not
they form bases for those spaces.





%%%%%%%%%%%%%%%%%%%%%%%%%%%%%%%%%%%%%%%%%%%%%%%%%%%%%%%%%%%%%%%%%%%%

\chapter{Classical Harmonic Analysis}\label{cha}

%%%%%%%%%%%%%%%%%%%%%%%%%%%%%%%%%%%%%%%%%%%%%%%%%%%%%%%%%%%%%%%%%%%%




We now
leave the broad discussion about bases and conditionality in Banach spaces and
instead focus our attention to some particular classes of the $L^p$ Banach
spaces. The theory of classical harmonic analysis provides a good starting point
from which to tackle problems relating to bases and conditionality in this 
setting.

In this chapter we give an overview of some basic results from classical
harmonic analysis and Fourier theory.
We begin by introducing the fundamental concepts of harmonic analysis on locally
compact Abelian groups. The Fourier transform and Fourier multipliers will be
defined. Connections between multiplier transforms and convolution operators 
will
be drawn in Section \ref{lca}, as well as important examples given. In
Section \ref{lca} we discuss some of the classical results of M. Riesz,
Littlewood--Paley and Marcinkiewicz.

The general theory given in this chapter can be used to consider the
the set of functions $\{\varphi_n\}_{n\in\Z}\subset L^p(\T)$, where
$\varphi_n:t\mapsto e^{int}$. Is this a basis for the Banach space $L^p(\T)$?
Is it an unconditional basic sequence? If not, how close is it
to being an unconditional sequence? The results stated in Section 
\ref{lca}
answer such questions.




%%%%%%%%%%%%%%%%
\section{Harmonic Analysis on Locally Compact Abelian Groups}\label{lca}
%%%%%%%%%%%%%%%%


In this section we give the necessary background from which we can begin to
answer
the questions from
above. The theory mentioned here can be found in \cite{Katznelson}
and is quite 
standard. In what follows, if $G$ is a group, the inverse of an
element $x\in G$ will be denoted by $-x$, and the group operation from
$G\times G$ into $G$ will be denoted by $(x,y)\mapsto x+y$. 


\begin{definition} A {\em locally compact Abelian group} (or an LCA group)
$G$ is an Abelian group
which is also a locally compact Hausdorff space such that the group operations
$x\mapsto -x$ from $G$ onto $G$ and $(x,y)\mapsto x+y$ from $G\times G$
onto $G$ are continuous. 
\end{definition}

The most-studied examples of LCA groups are the integers $\Z$, the circle group
$\T$ and real line $\R$ with their usual topologies. 
It is easy to verify that any Abelian group can be made into an LCA group
when endowed with the discrete topology. In this thesis the circle group $\T$
will feature most often. It can be modelled by the unit circle
$\{\omega\in\C:|\omega|=1\}$
in the complex plane, or as the quotient group $\R/2\pi\Z$. We shall usually
adopt the latter model,
thinking of $\T$ as the interval $[0,2\pi]$ with endpoints identified and
addition as the group operation.

A good reason to study LCA groups is
that we can use them to construct measure spaces that are translation invariant.

\begin{definition} Let $G$ be a locally compact Abelian group. A {\em Haar
measure} on $G$ is a positive regular Borel measure $\mu$ having the
following two properties:

(i) $\mu(E)<\infty$ if $E\subseteq G$ is compact; and

(ii) $\mu(E+x)=\mu(E)$ for all measurable $E\subseteq G$ and all $x\in G$.
\end{definition}

\begin{theorem}\cite[Chapter VII, \S 2]{Katznelson}
Let $G$ be an LCA group. Then a Haar measure on $G$ exists and is unique up to
multiplication of a positive constant.
\end{theorem}

Hence one often speaks of {\em the} Haar measure. For $G=\T$, Haar measure is
usually taken to be normalised
Lebesgue measure, $(2\pi)^{-1}dt$. If $G$ is discrete and infinite, Haar measure
is usually
normalised to have mass one at each point. We denote the $L^p$ space of
functions on $G$ with respect to Haar measure by $L^p(G)$. As mentioned in
Example \reff{Banach Spaces}{L^p example}, we will not usually distinguish
between functions defined on $G$ and their $L^p$ equivalence classes.

Using Haar measure, we may turn $L^1(G)$ into a Banach algebra.

\begin{theorem}\label{convolution}
Let $G$ be an LCA group with Haar measure $dy$
and suppose $f,g\in L^1(G)$. Then for almost all $x\in G$, the function
$y\mapsto f(x-y)g(y)$ for $y\in G$ is integrable on $G$.
If we write 
\[h(x)=\int_G f(x-y)g(y)dy,\]
then $h\in L^1(G)$ and $\norm{h}_1\leq\norm{f}_1\norm{g}_1$.
\end{theorem}

\begin{definition} Let $G,f,g$ be as in Theorem \reff{lca}{convolution}. Then
the {\em convolution} of $f$ and $g$, denoted $f*g$, is given by
\[(f*g)(x)=\int_G f(x-y)g(y)dy\]
for almost all $x\in G$
\end{definition}

\begin{corollary} Let $G$ be an LCA group. Then $L^1(G)$ is a
Banach Algebra under convolution and pointwise addition.
\end{corollary}

Our next aim is to define the Fourier transform of a function in $L^1(G)$.
We start by introducing the characters of $G$.

\begin{definition} A {\em character} on an LCA group $G$ is a continuous mapping
$\xi$ from $G$ into $\C$ such that $|\xi(x)|=1$
and $\xi(x+y)=\xi(x)\xi(y)$ for all $x,y\in G$. The set of all characters of
$G$ is denoted $\hatt{G}$.
\end{definition}

Thus a character is a continuous homomorphism of $G$ into $\T$. It can be shown
that the set of characters on $\T$ is the set $\{\varphi_n\}_{n\in\Z}$, where
$\varphi_n(t)=e^{int}$ for all $n\in\Z$ and all $t\in[0,2\pi]$. Every character
of
$\Z$ has the form $n\mapsto e^{int}$ for some $t\in[0,2\pi]$. The characters of
$\R$ are all of the form $x\mapsto e^{ixy}$ for some $y\in\R$.

For any LCA group $G$, the set of characters $\hatt{G}$ can be made 
an Abelian group. We shall denote its group operation by $+$ (in what follows, 
this should cause no confusion with the group operation $+$ for $G$) and
define it by $(\xi_1+\xi_2)(x)=\xi_1(x)\xi_2(x)$ for all $x\in G$.
We write $\ip<x,\xi>$ for $\xi(x)$.

A topology is defined on $\hatt{G}$ by specifying that
$\{\xi_n\}_{n=1}^{\infty}\subset\hatt{G}$ converges to $\xi\in\hatt{G}$ if
$\{\xi_n\}_{n=1}^{\infty}\subset\hatt{G}$ converges uniformly to $\xi$ on all
compact subsets of $G$. It can be shown that this topology turns $\hatt{G}$ into
a locally compact Abelian group (see \cite[Chapter VII, \S 3]{Katznelson}).
We say that $\hatt{G}$ is the {\em dual group}
of $G$.

The {\em Pontryagin Duality Theorem} (see \cite[p.189]{Katznelson}) asserts
that, for $x\in G$
and $\xi\in\hatt{G}$, the mapping $\xi\mapsto \ip<x,\xi>$ is a character on
$\hatt{G}$, and every character on $\hatt{G}$ is of this form. Moreover, the
topology induced by uniform convergence on compact subsets of $\hatt{G}$
coincides with the original topology on $G$. In otherwords, if $\hatt{G}$ is the
dual group of $G$, then $G$ is the dual of $\hatt{G}$.

We illustrate the Pontryagin Duality Theorem for the LCA group $\T$.
We saw that there exists a
bijection between $\hatt{\T}$ and $\Z$, since given any $\xi\in\hatt{\T}$, there
is a unique $n\in\Z$ such that $\ip<t,\xi>=e^{int}$ for all $t\in\T$.
Moreover, the topology $\hatt{T}$ is the
discrete topology. Thus $\hatt{\T}\simeq\Z$. One may similarly show that
$\hatt{\Z}\simeq\T$. This duality may be described (by abuse of notation) as
$\ip<n,t>=e^{int}=\ip<t,n>$ for
all $t\in\T$ and all $n\in\Z$. (On
the left hand side, we regard $n$ as an element of $\Z$ while on the right
hand side we regard it as a function on $\T$.)

We now have all the tools needed to define the Fourier transform of an 
integrable
function on an LCA group.

\begin{definition}\label{Fourier transf}
Let $G$ be an LCA group. Then the {\em
Fourier transform} of a function $f\in L^1(G)$ is defined by
\[\hatt{f}(\xi) = \int_G \overline{\ip<y,\xi>}f(y)dy\]
for all $\xi\in\hatt{G}$, where $dy$ is Haar measure on $G$.
\end{definition}

Thus the Fourier transform of a function $f\in L^1(\Z)$ is
\[\hatt{f}(t)=\sum_{n=-\infty}^{\infty}e^{-int}f(n)\]
for all $t\in\T$, since Haar measure on $\Z$ is unit point-mass measure.
Similarly, the Fourier transform of a function $f\in L^1(\T)$ is
\[\hatt{f}(n) = \frac{1}{2\pi}\int_0^{2\pi}e^{-int}f(t)dt\]
for all $n\in\Z$, where $dt$ is Lebesgue measure. The {\em Fourier series} of
$f$ is the corresponding formal expression
\[\sum_{n=-\infty}^{\infty}\hatt{f}(n)\varphi_n\]
where $\varphi_n(t)=e^{int}$ for all $t\in\T$ and all $n\in\Z$.


We aim to use the Fourier transform to construct a large class of operators
which act on the $L^p$ space of a given LCA group. We may do this via {\em
Plancherel's Theorem}.

\begin{theorem}\label{isometry}\cite[Chapter VII, \S 4]{Katznelson}
{\em Plancherel's Theorem.} Let $G$ be a locally compact Abelian group. Then the
Fourier transform on $L^1(G)$ is an isometry of $L^1\cap L^2(G)$ onto a dense
subspace of $L^2(\hatt{G})$. Hence it can be extended to an isometry of $L^2(G)$
onto $L^2(\hatt{G})$.
\end{theorem}

For the circle group $\T$, Plancherel's theorem takes the following form.

\begin{theorem}\label{Plancherel for T}\cite[Theorem I.5.5]{Katznelson}

(i) Let $f\in L^2(\T)$. Then
\[\sum_{n=\infty}^{\infty}|\hatt{f}(n)|^2=
\frac{1}{2\pi}\int_0^{2\pi}|f(t)|^2\,dt,\]
that is, $\ssnorm{\hatt{f}}_{L^2(\Z)}=\norm{f}_{L^2(\T)}$.

(ii) For each $f\in L^2(\T)$,
\[f=\lim_{N\rightarrow\infty}\sum_{n=-N}^N\hatt{f}_n\varphi_n\]
in $L^2(\T)$ norm, where $\varphi_n:t\mapsto e^{int}$.

(iii) Given any sequence $\{a_n\}_{n=-\infty}^{\infty}$ of complex numbers in
$L^2(\Z)$, then there exists a unique $f\in L^2(\T)$ such that
$a_n=\hatt{f}(n)$.

Thus the correspondence $f\leftrightarrow\{\hatt{f}(n)\}_{n=-\infty}^{\infty}$
is an isometry between $L^2(\T)$ and $L^2(\Z)$.
\end{theorem}


A by-product of the above theorem is that the set $\{\varphi_n\}_{n\in\Z}$ of
functions in the Hilbert space $L^2(\T)$ forms a complete orthonormal system 
(and
hence an unconditional basis) with respect to the inner product
\[\ip<f,g>=\frac{1}{2\pi}\int_0^{2\pi}f(t)\overline{g(t)}\,dt.\]

It is harder to establish whether or not $\{\varphi_n\}_{n\in\Z}$ is a basis for
$L^p(\T)$ when $1\leq p<\infty$ and $p\neq2$. Standard results about Fej\'{e}r's
kernel (see Section \ref{lca} and \cite[I.2.6]{Katznelson}) give the following
facts. First recall that a {\em trigonometric polynomial} is a function on 
$\T$ of 
the form $\sum_{n=-N}^Na_n\varphi_n$ where $N\in\N$ and $a_n\in\C$.

\begin{theorem}\label{Fejer facts}
Fix $1\leq p<\infty$. Then

(i) the trigonometric polynomials are dense in $L^p(\T)$;

(ii) if $f,g\in L^p(\T)$ and $\hatt{f}(n)=\hatt{g}(n)$ for all $n\in\Z$,
then $f=g$; and


(iii) if the Fourier series of a function $f\in L^p(\T)$ does converge in
$L^p$-norm, then it converges to $f$.
\end{theorem}

So if the Fourier series does converge for each function in $L^p(\T)$,
$\{\varphi_n\}_{n\in\Z}$ is a basis for $L^p(\T)$. Otherwise,
one would like to know whether or not $\{\varphi_n\}_{n\in\Z}$ is a basic 
sequence in $L^p(\T)$ for $p\neq2$, and whether this sequence is unconditional.
One way to study such problems is through multiplier transforms.

\begin{definition}\label{multipliers}
Let $G$ be an LCA group and let
$\phi:\hatt{G}\rightarrow\C$ be bounded and measurable on $\hatt{G}$. By
Plancherel's Theorem, define $S_{\phi}$ to be the continuous linear mapping of
$L^2(G)$ into itself for which
$({S_{\phi}f})\,\hatt{\,} = \phi \hatt{f}$ for all $f\in L^2(G)$.
Then for $1\leq p\leq\infty$, $\phi$ is said to be an 
{\em $L^p(G)$ multiplier} if
and only if there is a constant $C>0$ such that
\begin{equation}\label{eq-multiplier norm}
\norm{S_{\phi}f}_p\leq C_p\norm{f}_p
\end{equation}
for all $f\in L^2\cap L^p(\T)$.
In this case $S_{\phi}$ is called the {\em multiplier transform corresponding
to $\phi$ on $L^p(G)$} and $\phi$ may also be referred to as a
{\em Fourier multiplier} or a {\em multiplier function} for $L^p(G)$.
\end{definition}


The space of Fourier multipliers for $L^p(G)$ will be denoted by
$M_p(\hatt{G})$. We turn $M_p(\hatt{G})$ into a Banach space by equipping it 
with
the following norm: for $\phi\in M_p(\hatt{G})$ define
$\norm{\phi}_{M_p(\hatt{G})}$ to be the usual operator norm on $L^p(G)$ for the
operator $S_{\phi}$. Thus $\norm{\phi}_{M_p(\hatt{G})}$ is the smallest
possible $C\geq 0$ for which (\ref{eq-multiplier norm}) holds. It is
trivial to check from the definitions that if $\phi_1,\phi_2\in M_p(\hatt{G})$,
then the product $\phi_1\phi_2$, defined by pointwise operations, is in
$M_p(\hatt{G})$ and
\[\norm{\phi_1\phi_2}_{M_p(\hatt{G})}\leq\norm{\phi_1}_{M_p(\hatt{G})}
\norm{\phi_2}_{M_p(\hatt{G})}.\]
In fact, we have the following result.

\begin{proposition}\label{multiplier algebra}\cite[\S3]{BG Spectral}
Let $G$ be an LCA group. Then for $1\leq p<\infty$,
$M_p(\hatt{G})$ is a Banach algebra under pointwise operations, and the mapping
$\phi\mapsto S_{\phi}$ is an identity-preserving algebra homomorphism of
$M_p(\hatt{G})$ into $\B(L^p(G))$.
\end{proposition}

The next example illustrates some of the concepts that
have been introduced so far.

\begin{example}\label{lca example}
Let $G=\T$ and let $\varphi_n:t\mapsto e^{int}$ for all $t\in\T$ and $n\in\Z$.
Fix an $m\in\Z$ 
and consider the characteristic function $\chi_{\{m\}}$ on $\Z$.
Then for any $f\in L^2(\T)$ we have, by Definition \reff{lca}{multipliers},
\[\bigl(S_{\chi_{\{m\}}}f\bigr)\,\hatt{\,}\,(n)=\chi_{\{m\}}(n)\hatt{f}(n)
=\chi_{\{m\}}(n)\hatt{f}(m)=\hatt{\varphi}_m(n)\hatt{f}=
\bigl(\hatt{f}(m)\varphi_m\bigr)\,\hatt{\,}\,(n)\]
for all $n\in\Z$. Thus $S_{\chi_{\{m\}}}f=\hatt{f}(m)\varphi_m$, and
$S_{\chi_{\{m\}}}$ projects each $f$ onto the $m$th term of its Fourier
series.

Now we consider the convolution product of $f$ with $\varphi_m$.
For almost all $t\in\T$ we have
\begin{eqnarray*}
(\varphi_m*f)(t) &=& \frac{1}{2\pi}\int_0^{2\pi}\varphi_m(t-s)f(s)ds \\
& = & e^{imt}\,\frac{1}{2\pi}\int_0^{2\pi}e^{-ims}f(s)ds \\
& = & e^{imt}\hatt{f}(m).
\end{eqnarray*}
Thus $S_{\chi_{\{m\}}}$ and the operator $f\mapsto\varphi_m*f$ coincide on
$L^2(\T)$. It is easy to see that $\ssnorm{S_{\chi_{\{m\}}}}=1$ and hence
$\ssnorm{\chi_{\{m\}}}_{M_p(\hatt{G})}=1$.
\end{example}

We now give one reason why multiplier transforms are of interest to us.
Let $\varphi_n(t)=e^{int}$ for each $n\in\Z$ and $t\in\T$. One might hope that
$\{\varphi_n\}_{n\in\Z}$ is a basis for $L^p(\T)$. Assume for the moment that 
this is the case. The expansion of a
function $f\in L^p(\T)$ with respect to this basis is then given by
$\sum_{n\in\Z}\hatt{f}(n)\varphi_n$. 
If one wanted to prove that the basis was unconditional, it would suffice to
find (by Proposition \reff{bases}{unconditional conv}) a constant $C_p$
depending only on $p$ such that
\[\snorm{\sum_{n\in\Z}\varepsilon_n\hatt{f}(n)\varphi_n}_p\leq C_p\norm{f}_p\]
for all $f\in L^p(\T)$ and all choices
$\varepsilon_n=\pm 1$. This would be
equivalent to finding a constant $C_p$ such that
\[\norm{S_{\varepsilon}f}_p\leq C_p\norm{f}_p\]
for all $f\in L^2\cap L^p(\T)$ and all functions $\varepsilon:\Z\rightarrow\C$
taking values in $\{-1,1\}$. Motivated by this, we develop the general theory of
multipliers a little further in the next section.


\chapter{Conclusion}\label{ccl}

In mathematics, certain kinds of mistaken proof are often exhibited, and sometimes collected, as illustrations of a concept of mathematical fallacy. There is a distinction between a simple mistake and a mathematical fallacy in a proof: a mistake in a proof leads to an invalid proof just in the same way, but in the best-known examples of mathematical fallacies, there is some concealment in the presentation of the proof. For example, the reason validity fails may be a division by zero that is hidden by algebraic notation. There is a striking quality of the mathematical fallacy: as typically presented, it leads not only to an absurd result, but does so in a crafty or clever way. Therefore these fallacies, for pedagogic reasons, usually take the form of spurious proofs of obvious contradictions. Although the proofs are flawed, the errors, usually by design, are comparatively subtle, or designed to show that certain steps are conditional, and should not be applied in the cases that are the exceptions to the rules. \\

\noindent The traditional way of presenting a mathematical fallacy is to give an invalid step of deduction mixed in with valid steps, so that the meaning of fallacy is here slightly different from the logical fallacy. The latter applies normally to a form of argument that is not a genuine rule of logic, where the problematic mathematical step is typically a correct rule applied with a tacit wrong assumption. Beyond pedagogy, the resolution of a fallacy can lead to deeper insights into a subject (such as the introduction of Pasch's axiom of Euclidean geometry and the five color theorem of graph theory). Pseudaria, an ancient lost book of false proofs, is attributed to Euclid. \\

\noindent Mathematical fallacies exist in many branches of mathematics. In elementary algebra, typical examples may involve a step where division by zero is performed, where a root is incorrectly extracted or, more generally, where different values of a multiple valued function are equated. Well-known fallacies also exist in elementary Euclidean geometry and calculus.







%%%%%%%%%%%%%%%%%%%%%%%%%%%%%%%%%%%%%%%%%%%%%%%%%%%%%%%%%%%%%%%%%%%%%%%%%%

\clearpage
\addcontentsline{toc}{chapter}{References}
\bibliographystyle{unswthesis}

\begin{thebibliography}{999}

\bibitem
{BBG} Benzinger, H., Berkson, E. and Gillespie, T.A.,
Spectral families of projections, semigroups, and differential operators,
\textit{Tran. Amer. Math. Soc.} \textbf{275} (1983), 431--475.

\bibitem
{BG Fourier} Berkson, E. and Gillespie, T.A.,
Fourier series criteria for operator decomposability,
\textit{Integral Equations Operator Theory} \textbf{9} (1986), 767--789.

\bibitem
{BG Spectral} Berkson, E., and Gillespie, T.A.,
Spectral decompositions and harmonic analysis on UMD spaces,
\textit{Studia Mathematica} \textbf{112}(1) (1994), 12--49.

\bibitem
{BGM} Berkson, E. Gillespie, T.A. and Muhly, P.S.,
Abstract spectral decompositions guaranteed by the Hilbert transform,
\textit{Proc. London Math. Soc. (3)} \textbf{53} (1986), 489--517.

\bibitem
{Bourgain telaviv} 
Bourgain, J., {\em Martingale transforms and geometry of Banach spaces},
Israel seminar on geometrical aspects of functional analysis
(1983/84), XIV, 16 pp., Tel Aviv Univ., Tel Aviv, 1984.

\bibitem
{Bourgain} Bourgain, J.,
Some remarks on Banach spaces in which martingale difference sequences are
unconditional,
\textit{Ark. Mat.} \textbf{21} (1983), 163--168.

\bibitem
{Bourg} Bourgain, J.,
Vector-valued singular integrals and the $H^1$-BMO duality,
\textit{Probability Theory and Harmonic Analysis} (ed. Chao, J. A.) (1986), 
1--19.

\bibitem
{Burk3} Burkholder, D.L.,
A geometrical characterisation of Banach spaces in which martingale difference
sequences are unconditional,
\textit{Ann. Probability} \textbf{9} (1981), 997--1011.

\bibitem
{Burk2} Burkholder, D.L.,
A geometric condition that implies the existence of certain singular integrals
of Banach-space valued functions,
\textit{Proceedings of Conference on Harmonic Analysis in Honor of Antoni
Zygmund} (Chicago, Illinois, 1981), Wadsworth Publishers: Belmont,
1983, pp. 270--286.

\bibitem
{Burk4} Burkholder, D.L.,
Martingales and Fourier analysis in Banach spaces,
\textit{Lecture Notes in Mathematics} \textbf{1206} (1986), 61--108.

\bibitem
{Burk1} Burkholder, D.L.,
Martingale transforms,
\textit{Ann. Math. Statist.} \textbf{37} (1966), 1494--1504.

\bibitem
{Coifman} Coifman R.R., and Weiss, G.,
\textit{Transference methods in analysis},
Regional Conference Series in Mathematics 31,
Amer. Math. Soc., Providence, R.I., 1977.

\bibitem
{Con} Conway, J.,
\textit{A Course in Functional Analysis},
Springer-Verlag: New York, 1985.

\bibitem
{Dowson} Dowson, H.,
\textit{Spectral theory of linear operators},
London Math. Soc. Monographs, No. 12. Academic Press: New York, 1978.

\bibitem
{Doust} 
Doust, I., Norms of $0$-$1$ matrices in $\CC_p$,  pp 50-55, Proc.
Centre Math. Appl. Austral. Nat. Univ., 39, Austral. Nat. Univ.,
Canberra, 2000.

\bibitem
{DG} Doust, I.  and Gillespie, T. A.,
Schur multipliers on $\CC_p$ spaces,
in preparation.

\bibitem
{Dun} Dunford, N., and Schwartz, J.T.
\textit{Linear operators I: General theory},
Pure and Applied Mathematics 7. Interscience: New York, 1958.

\bibitem
{Gaudry} Edwards, R.E., and Gaudry, G.I.,
\textit{Littlewood--Paley and Multiplier Theory},
Springer-Verlag: Berlin, 1977.

\bibitem
{Katznelson}
Katznelson, Y.,
\textit{An introduction to harmonic analysis},
Second corrected edition, Dover: New York, 1976.

\bibitem
{Gohberg}
Gohberg, I.C., and Kre\u{\i}n, M. G.,
\textit{Introduction to the theory of linear nonselfadjoint operators},
Translations of Mathematical Monographs, Vol. 18, Amer. Math. Soc.,
Providence, R.I., 1969.

\bibitem
{Gowers} Gowers, W. T., and Maurey, B.,
The unconditional basic sequence problem,
\textit{J. Amer. Math. Soc.} \textbf{6} (1993), 851--874.

\bibitem
{Lind}
Lindenstrauss, J., and Tzafriri, L.,
\textit{Classical Banach Spaces I and II},
Springer: Berlin, 1996.

\bibitem
{Pedersen}
Pedersen, G. K.,
\textit{Analysis Now},
Springer: New York, 1989.

\bibitem
{Stein1}
Stein, E. M.,
The development of square functions in the work of A. Zygmund,
\textit{Bulletin (New Series) of the Amer. Math. Soc},
\textbf{6} (1982), 5--30.

\bibitem
{Stein}
Stein, E. M.,
\textit{Topics in Harmonic Analysis Related to the Littlewood--Paley Theory},
Princeton UP: Princeton, 1970.

\end{thebibliography}




\end{document}





